\documentclass{article}

\usepackage[utf8]{inputenc}

\usepackage[T1]{fontenc}
%\usepackage[scale=0.85]{FiraMono}
%\usepackage[sfdefault,scaled=.85]{FiraSans}
\usepackage[math]{iwona}

%\usepackage[polish]{babel}
\usepackage{polski}

\usepackage{xspace}
\usepackage{amsmath}

\newcommand{\devc}{\textit{DevContainers}\xspace}
\newcommand{\vsc}{\textit{VS Code}\xspace}

\author{Piotr Zawadzki}
\title{Kontenery i \LaTeX}
\begin{document}
\maketitle{}

\tableofcontents{}

\section{Wprowadzenie}

ąćęńóśźż

ĄĆĘŃÓŚŹŻ

\devc{} to zaimplementowany w \vsc{} mechanizm dynamicznego tworzenia kontenerów Docker'a zawierających narzędzia potrzebne do pracy nad danym projektem.
W katalogu \texttt{.devcontainer} znajduje się plik
\texttt{devcontainer.json} zawierający przepis na utworzenie i~konfigurację kontenera.

\subsection{Wykorzystanie mechanizmu \devc{} do składu dokumentów}

\LaTeX{} to bardzo złożony system składu dokumentów, składający się z wielu współdziałających ze sobą aplikacji.
Instalacja systemu jest dość skomplikowana, stąd też jest on rozprowadzany w~tzw.~dystrybucjach zawierających dostrojone do siebie aplikacje, z których dwie najbardziej popularne to \TeX{}Live i Mik\TeX{}.
Sam proces składu dokumentu bardzo przypomina kompilację programu, stąd pomysł, aby środowisko składu przygotować w~ramach mechanizmu \devc{}.

\subsection{Dwa}

Niniejszy projekt jest praktyczną realizacją pomysłu.
Do składu wykorzystywany jest kontener zbudowany na bazie \textsc{Ubuntu 22.04} (dystrybucja \TeX{}Live z 2021).

\section{Skład dokumentu}

Zapraszam do wypróbowania działania osiągniętego w ten sposób systemu składu.
Dokument źródłowy wprowadzamy w \vsc{}.
W procesie składu na jego podstawie generowany jest plik PDF, który można wyświetlić w oknie podglądu.
\begin{align}
  a & = 4+2x+\sin(y)         \\
  b & = 28 -3 x + \cos\alpha
\end{align}

Tu mamy listę wyliczaną
\begin{itemize}
  \item jeden,
  \item dwa,
  \item trzy
\end{itemize}

\subsection{Dodatki i udogodnienia}

Zapraszam do wypróbowania działania osiągniętego w ten sposób systemu składu.
Dokument źródłowy wprowadzamy w \vsc{}.
W procesie składu na jego podstawie generowany jest plik PDF, który można wyświetlić w oknie podglądu.
\begin{align}
  a & = 4+2x+\sin(y)         \\
  b & = 28 -3 x + \cos\alpha
\end{align}

Tu mamy listę wyliczaną
\begin{itemize}
  \item jeden,
  \item dwa,
  \item trzy
\end{itemize}

\subsection{Dostosowanie zawartości kontenera do wymogu chwili}

Zapraszam do wypróbowania działania osiągniętego w ten sposób systemu składu.
Dokument źródłowy wprowadzamy w \vsc{}.
W procesie składu na jego podstawie generowany jest plik PDF, który można wyświetlić w oknie podglądu.
\begin{align}
  a & = 4+2x+\sin(y)         \\
  b & = 28 -3 x + \cos\alpha
\end{align}

Tu mamy listę wyliczaną
\begin{itemize}
  \item jeden,
  \item dwa,
  \item trzy
\end{itemize}

\end{document}
