\documentclass{article}

\usepackage[utf8]{inputenc}

\usepackage[T1]{fontenc}
%\usepackage[scale=0.85]{FiraMono}
%\usepackage[sfdefault,scaled=.85]{FiraSans}
\usepackage[math]{iwona}

%\usepackage[polish]{babel}
\usepackage{polski}
\usepackage{lipsum}

\usepackage{amsmath}


\author{Piotr Zawadzki}
\title{Kontenery i \LaTeX}
\begin{document}
\maketitle{}

\tableofcontents{}

\section{Pierwszy}

Witaj świecie. To przykładowy dokument ilustrujący skład dokumentu w~języku polskim:
\begin{itemize}
  \item jeden,
  \item dwa,
  \item trzy
\end{itemize}
Do powyższego wyliczenia dodajmy jeszcze wzór:
\begin{align}
  a & = 4+2x+\sin(y) \,,         \\
  b & = 28 -3 x + \cos\alpha \,.
\end{align}

\subsection{Jeden} \lipsum{}

\subsection{Dwa} \lipsum{}

\section{Drugi} \lipsum{} \lipsum{}

\subsection{Jeden} \lipsum{}

\subsection{Dwa} \lipsum{}

\end{document}
